\documentclass[main.tex]{subfiles}

\begin{document}

\section{Filtro Adaptativo}
Al momento de elegir el algoritmo que se implementó para el filtro, 
se analizaron varias alternativas, entre ellas LMS, NLMS, VS-LMS y Sign LMS.
En primer lugar, los algoritmos Sign LMS, entre ellos, sign-error, sign-data
y sign-sign fueron descartados ya que el baud rate de la señal era de 250bps,
cabe recordar que esta variante de LMS es de utilidad para ecualizar canales de
comunicación digital de alta velocidad. En segundo lugar y  luego de analizar 
los resultados obtenidos y las conclusiones propuestas por Bismor \cite{bismor},
se decidió no implementar VS-LMS. En las palabras de los autores: 
"no hay algoritmo VS-LMS que sea tan versátil, fácil de implementar y adecuado 
para aplicaciones en tiempo real como el NLMS."\newline
Esta observación, si bien descarta la implementación de VS-LMS, 
plantea un último debate respecto de si corresponde utilizar LMS o NLMS. 

\subsection{LMS}
\subsection{NLMS}

\end{document}